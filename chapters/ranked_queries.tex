Ranked queries are queries concerned with the ranking of objects. With ranked queries, inputs and outputs may be objects that are ordered by a scoring function.

\section{Concepts}

Given a set of objects \(R\), a top-k query returns an ordered set of \(k\) objects of \(R\) with the greatest ranking. The ranking is defined by a scoring function \(q\) that assigns a score to each object in \(R\). All elements in the result set \(T\) must have a score greater than or equal to the score of the elements that are not in the result set. More formally,

\[
  \forall_{r \in T}
  \forall_{s \in R - T} :
  \left(q(r) \geq q(s) \right).
\]

Given two sets of objects \(R\) and \(S\), the top-k join returns an ordered set of \(k\) join results with the greatest ranking. It is a special case of the top-k query that operates on the results of a join. Join results are \((r, s)\) tuples of \(R \times S\) that satisfy a join predicate \(\phi\). The ranking is similarly defined by a score function \(q\) that assigns a score to each tuple of the join result. All elements in the result must have a score greater than or equal to the score of the elements that are not in the result set.

A score function assigns a score to each object in a set to define a ranking. The score may be an aggregation of multiple object score attributes \(a_1, a_2, a_3, \dotsc\) by an aggregate function \(\gamma\). For instance, if \(\gamma\) is SUM, the score of an object is the sum of the score attributes of the object. \(\gamma\) is considered a monotone aggregate function if
\[
  \forall_{(x, y)} \forall_{(x', y')} :
  \left(
  x \leq x'
  \land
  y \leq y'
  \right)
  \Rightarrow
  \left(
  \gamma(x, y) \leq \gamma(x', y')
  \right),
\]
SUM, MAX and MIN are all examples of monotone aggregate functions.

Using monotone aggregate functions for scores, assertions can be made to speed up the evaluation of ranked queries. A key insight that is that if the score is a monotone aggregation of attributes \(a_1\) and \(a_2\), the aggregation of the maximum value of \(a_1\) and the maximum value of \(a_2\) for a set of elements forms an upper bound for the score of all elements in the set. More formally, given a set of elements \(R\) with attributes \(a_1\) and \(a_2\),
\[
  \forall r \in R :
  \gamma(r.a_1, r.a_2)
  \leq
  \gamma(\max(s.a_1 \mid s \in R), \max(s.a_2 \mid s \in R))
\]
where \(\max(s.a_1 \mid s \in R)\) and \(\max(s.a_2 \mid \in R)\) are the maximum values of \(a_1\) and \(a_2\) for \(R\). If it can be asserted that all elements in a result must have a score greater than \(s_{\min}\), then knowing the maximum value of each attribute for a set of elements is enough to determine if any elements in the set can be part of the result by comparing the upper bound with \(s_{\min}\). If the elements of the set cannot be part of the result, there would be no need to evaluate each element for inclusion in the result by computing the score and comparing it with \(s_{\min}\).

For top-k joins, the join inputs \(R\) and \(S\) may be individually ranked by score functions \(q_R\) and \(q_S\) respectively. The scores assigned by \(q_R\) and \(q_S\) can be considered as attributes of each joined tuple. Then the score of a joined tuple can be an aggregation of the scores assigned by \(q_R\) and \(q_S\). In this case, the score of a joined tuple would be \(q(r, s) = \gamma(q_R(r), q_S(s))\) where \(\gamma\) is an aggregate function.

For top-k joins where the score is a monotone aggregation of scores, the aggregation of the score of each highest ranked element in ranked sets \(R\) and \(S\) forms an upper bound for the score of all pairs in \(R \times S\). More formally, given ranked sets \(R\) and \(S\),
\[
  \forall (r, s) \in R \times S :
  \gamma(q_R(r), q_S(s))
  \leq
  \gamma(q_R(R_{\max}), q_S(S_{\max}))
\]
where \(R_{\max}\) and \(S_{\max}\) are the highest ranked elements of their respective sets. If it can be asserted that all elements in a join result must have a score greater than \(s_{\min}\), then knowing the maximum scores of \(R\) and \(S\) is enough to determine if any joined elements of \(R\) and \(S\) can be part of the join result. If the results of the join cannot be part of the result, there would be no need to evaluate each element of the join for inclusion in the result by computing the score and comparing it with \(s_{\min}\). More importantly, there would be no need to perform a join at all.

\section{Ranked range search}

Spatial queries may be augmented to require rank-based filtering or sorting of results. A basic method to achieve this would be a two-step process of performing the spatial query first, then filtering or sorting the outputs second. However, the R-tree range search algorithm can be modified to efficiently achieve both rank-based filtering and sorting in a single step. The modified algorithm operates on a MAX aR-tree, so that each node contains the maximum score value of any object contained by the node. The MAX aggregate function has properties that allows both for filtering and sorting of query outputs using a modified range search algorithm.

The R-tree range search algorithm can be modified to only output objects with a score greater than some value \(s_{\min}\) by adding an additional condition to prune subtrees. The modification is based on the principle that any aR-tree node with MAX score lower than \(s_{\min}\) can be pruned from the search because the subtree cannot contain any entries that satisfy the rank condition. This is the same principle that allows pruning subtrees from the search for not intersecting with the query box because the subtree cannot contain any entries that intersect with the query box.

The R-tree range search algorithm can be modified to output results in order of descending rank by dequeuing queue elements in order of descending upper score bound. The modification is based on the principle that in the queue of a range search, the highest ranked R-tree entry that has not yet been visited must be found via the highest ranked queue entry. Queue entries are ranked by the MAX score for inner node entries, and actual score for leaf node entries. The highest ranked unvisited R-tree entry can be found in the queue either as an ancestor of an inner node with the greatest MAX score, or as the leaf node entry with the greatest score. By always dequeuing the highest ranked queue entry, queue entries are processed in best-first order which guarantees that leaf entries will be visited in descending rank order.

\section{Top-k joins}

A basic method to perform a ranked join is performing a join then sorting the results. The problem with this method is that it requires the join to be completely evaluated before any part of the output can be produced. Joins can be computationally expensive and can produce result sets that are unnecessarily large to answer a top-k join. A method that can produce the same ranked join outputs using only a partially evaluated join would be more efficient by avoiding the cost of the full join. Such a method requires efficient navigation of the join space and must also guarantee that the join results that have not been evaluated are not part of the ranked join results.

\subsection{Hash-based rank-join}

\com{Should perhaps elaborate on joins in general and ripple join here?}

An interesting alternative for ranked joins with monotone aggregate functions is accessing inputs in decreasing rank order and performing a ripple join. A ripple join allows incrementally finding partial join results by incrementally accessing inputs. Accessing inputs in decreasing rank order allows establishing an upper bound for the score of any results that have not yet been found. The upper bound \(T\), also called the threshold, is defined as
\[
  T = \max\{\gamma(R_{\min}, S_{\max}), \gamma(R_{\max}, S_{\min})\}
\]
where \(R_{\min}, R_{\max}, S_{\min}\) and \(S_{\max}\) are the minimum and maximum scores seen in \(R\) and \(S\) so far.
The intuitive explanation is that the next accessed element of \(R\) has at most the same score as the previous element, and may join with the element of \(S\) with the highest score, creating an aggregate score of at most \(\gamma(R_{\min}, S_{\max})\) and vice versa for the next accessed element of \(S\). At any point during processing, all known join results with a score above this threshold are guaranteed to be the first results of the join in order of descending rank, because any unknown join result must have a score below the threshold. The ripple join terminates when all join results have been produced by exhausting the inputs, but may be terminated earlier when enough results have been produced for a top-k join.

Ilyas et al.~\cite{ilyas2004supporting} proposed hash-based rank-join (HRJN), which is an operator that uses two ranked relational inputs to perform a ranked equijoin with a monotone aggregate function. Its key feature is that it incrementally produces outputs by evaluating the join in an order that optimizes for producing join results with high aggregate scores first. The algorithm accesses inputs in order of descending rank to incrementally evaluate the join and places join results in a heap ordered by aggregate score. It keeps track of the upper bound for the score of any joined tuple that has not yet been found. Accessing input elements causes the upper bound to decrease. When the element on top of the heap has a score greater than the upper bound, it can be removed and output as the next element in order of descending rank. A top-k query can be performed by using the \(k\) first outputs, which may not require evaluating the full join.

HRJN is an equijoin operator, meaning it joins entries by equality of a key, so that the join condition is \(\phi(r, s) : key_R(r) = key_S(s)\) where \(key_R\) and \(key_S\) get the keys of each entry. The ripple join is performed using a hash table per input to index accessed elements by their keys. Newly accessed elements are placed into their respective hash tables and joined with indexed elements from the other input by looking up the key in the other hash table. The algorithm can be made more generic with the realization that the use of hash tables as indexes is a particular choice for the key equality join predicate. The hash tables can be replaced by other indexes such as spatial indexes for other kinds of join predicates. HRJN is considered an instantiation of the Pull/Bound Rank Join framework~\cite{schnaitter2010optimal} and can be further generalized as part of the Score-First Paradigm~\cite{qi2015efficient}.

\subsection{Parallel top-k joins}

A critical difficulty with parallelizing HRJN is that it is inherently sequential. One theoretical approach is to process input elements in parallel. Hash table lookups could be performed in parallel, but incremental construction of the hash tables is not expected to parallelize well, especially coupled with concurrent reads and writes.

One way to achieve parallel top-k joins using sequential algorithms is using a divide-and-conquer strategy. The problem can be broken into \(n\) subproblems by splitting the input into \(n\) partitions and finding the top-k elements of each partition. The top-k elements of a set \(R\) is can be found by dividing \(R\) into \(n\) non-overlapping partitions \(R_1, R_2, \dotsc, R_n\), finding the top-k elements of each \(R_n\), concatenating the top-k elements of each partition, then finding the final top-k elements among the concatenated results. Representing a top-k query as a function \(top_k\) that returns the top-k result, it can be expressed as
\[
  top_k(R) = top_k(top_k(R_1) \cup top_k(R_2) \cup \cdots \cup top_k(R_n)).
\]

A top-k join on \(R\) and \(S\) can be similarly processed using a divide-and-conquer strategy by considering it as a top-k query on \(R \times S\). The total amount of elements in \(R \times S\) can be very large, so it is preferable to allow the generation of pairs within each subproblem instead of explicitly materializing and partitioning \(R \times S\). The solution is to partition \(R\) into \(n\) partitions \(R_1, R_2, \dotsc, R_n\) and \(S\) into \(m\) partitions \(S_1, S_2, \dotsc, S_m\). Each pair of partitions \(R_i\) and \(S_j\) creates a subproblem \(R_i \times S_j\). This creates \(nm\) total subproblems which in total covers the entirety of \(R \times S\).

Dividing the problem into subproblems and processing them in parallel is an opportunity to perform parallel top-k queries. The basic principle is that if each subproblem is given a smaller input, then each subproblem will require less work than the whole input. If each subproblem can be processed in parallel in less time than the whole problem, it should produce a result in less time, but may result in wasted work. Divide-and-conquer leads to overall wasted work for top-k joins~\cite{yu2010workload}. If the work required to process a two-way join without breaking into subproblems is \(w\), then breaking it into \(n\) subproblems requires \(\frac{1}{\sqrt{n}} w\) work per subproblem, for a total of \(\sqrt{n} w\) work. In other works, the total amount of work increases by a factor of the square root of the amount of subproblems.

It may be possible to avoid wasted work by estimating the score of the \(k^{th}\) element of the top-k result.

Kim et al.~\cite{kim2012parallel} proposed and evaluated methods to perform parallel top-k similarity joins using the MapReduce paradigm.

\section{Top-k spatial joins}

A top-k spatial join combines the aspects of a spatial join and a top-k join. Given two sets of spatial objects, a spatial predicate is used to join the sets, and the \(k\) highest ranked joined tuples form the result. Score-driven methods for top-k joins may not be able to properly utilize the spatial properties of the inputs to perform the join. Similarly, space-driven methods for spatial joins may not be able to properly utilize the ranked properties of the inputs to rank the output. This difference calls for specialized methods that can utilize both.

A basic approach is to use a spatial join as a basis to join the inputs, then ranking the join result to produce the top-k results. This approach can fully utilize the spatial properties of the inputs, but methods such as the basic R-tree join and plane sweep-based joins~\cite{arge1998scalable} do not have ways of prioritizing the join result computation according to the ranking. Therefore, the full spatial join has to be computed first, which may use more memory and processing time than necessary to produce the top-k result. Another approach is to apply the principles of HRJN by incrementally evaluating the spatial join until it can be determined that any joined tuple that has not yet been produced cannot be part of the top-k result. This would require an efficient method to evaluate partial spatial joins.

Qi et al.~\cite{qi2013efficient} proposed and evaluated several algorithms to perform top-k spatial distance joins on ranked inputs. All algorithms use aR-trees with MAX as the aggregate function in different ways. The Score-First Algorithm (SFA) is a variation of the HRJN algorithm that utilizes aR-trees instead of hash tables for indexing accessed input values. The Distance-First Algorithm (DFA) requires both inputs to be fully indexed by aR-trees and performs a ranked R-tree join. The Block-based Algorithm (BA) works similarly to the SFA by prioritizing objects with high scores first, but accesses inputs in blocks to bulk-load aR-trees and joins the aR-trees in a ripple join fashion.

\subsection{Ranked spatial join on MAX aR-trees}

As part of DFA, Qi et al.~\cite{qi2013efficient} described an algorithm can be used to perform a ranked spatial join on ranked inputs indexed by MAX aR-trees. If the score of a joined tuple \((r, s)\) is defined as \(q(r, s) = \gamma(q_R(r), q_S(s))\) where \(\gamma\) is a monotone aggregate function and \(q_R(r)\) and \(q_S(s)\) are the scores of the elements \(r\) and \(s\), then the upper bound for the score of all pairs of entries that can be produced by joining two nodes is the aggregate of the MAX score of each node. Similar to the ranked range search, the upper bound can be used both to filter and to output join results in descending ranked order.

The R-tree spatial join algorithm can be modified to only output join results with a score greater than some value \(s_{\min}\) using the same principles applied to the ranked range search algorithm. Any pair of nodes with MAX score aggregate lower than \(s_{\min}\) can be pruned from the search. This is because when joined, they cannot produce a pair of entries that satisfy the rank condition, just like a pair of nodes can be pruned from the search for not intersecting with each other.

The R-tree spatial join algorithm can be modified to output join results in order of descending rank by dequeuing queue elements in order of descending upper score bound. The modification is based on the principle that in the queue of a spatial join, the highest ranked pair of R-tree entries that has not yet been found must be found via the highest ranked queue entry. Queue entries are ranked by the MAX score aggregate for pairs of inner node entries, and actual score for pairs of leaf node entries. The highest ranked pair of R-tree entries can be found in the queue either as ancestors of a pair of inner nodes with the greatest MAX score, or as the pair of leaf node entries with the greatest score. By always dequeuing the highest ranked queue entry, queue entries are processed in best-first order which guarantees that pairs of leaf entries will be visited in descending rank order.

\subsection{Distance-First Algorithm}

The Distance-First Algorithm (DFA) is performed as a ranked spatial join on MAX aR-trees that terminates when the first \(k\) results are found. Each input must be fully indexed by a MAX aR-tree, either using a previously constructed index or by bulk loading an aR-tree prior to the join.

DFA works well if there is a correlation between the scores of the objects and their locations. Because the aR-tree entries are clustered by their locations and not their scores, pairs are pruned primarily due to the spatial predicate. The objects with the highest scores that are close to each other are identified fast, while the rest of the results may require expanding many pairs of nodes before it termination.

\subsection{Score-First Algorithm}

The Score-First Algorithm (SFA) is similar to HRJN in that it accesses inputs in order of descending rank to perform a ripple join, but instead of using hash tables to index accessed entries, it uses MAX aR-trees. This replaces the hash table insertion and probing operations with R-tree insertion and range search operations. When the algorithm has found at least \(k\) candidate results, the score of the \(k^{th}\) candidate result, called \(\theta\), is a threshold for the score of any results that have not yet been found that may replace the top-\(k\) candidates. SFA can use \(\theta\) to prune subtrees when probing the aR-tree to only find join results that will become top-\(k\) candidates.

With SFA, pairs of objects with high scores are found fast, and the results are expected to be found fast if the objects with the highest scores are close to each other. Otherwise, many entries have to be dynamically inserted and many range queries have to be performed if the objects among best pairs reside deep in the inputs. The overhead also increases with each accessed element as the cost of each insertion and query increases as the R-trees grow larger.

\subsection{Block-based Algorithm}

The Block-based Algorithm (BA) is similar to SFA by prioritizing objects with high scores, but instead of accessing one object at a time, it accesses inputs in blocks. An aR-tree is bulk loaded from each block. Each block is joined with all the blocks that have been accessed so far from the other input using aR-tree joins. BA keeps track of candidate top-k results, and can also use the score of the \(k^{th}\) candidate result to prune subtrees when performing each spatial join.

BA is an attempt to overcome SFA's failure to quickly find spatial join results, and DFA's failure to quickly find pairs with high scores. Performing joins at block level is more efficient than performing joins one entry at a time when considering the cost of processing per element. Indexing blocks with bulk loading is more efficient than dynamic insertion of each element, and joining aR-trees is more efficient than probing once per accessed element. The number of aR-tree joins per loaded block is limited by observing the upper bound for the score of any tuple that can be produced by joining a pair of aR-trees.
