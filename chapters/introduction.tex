In many applications, objects have both spatial attributes and non-spatial attributes. As an example, an application such as Google Maps provides points of interest and businesses, primarily with their geographic location, but also with reviews, user ratings and various other metadata. Spatial objects are also recorded in scientific fields such as atmospheric, oceanographic and environmental sciences with measurements of several attributes such as temperature, pressure and seismic activity.

Object attributes can be used as scores to derive a ranking, which often is used to perform a top-k query when the amount of objects is too large for all objects to be relevant. A top-k query retrieves the k objects with the best ranking. A top-k join is a special case of a top-k query which is performed on the results of a join. If the inputs of the join are each given a score, a top-k join can retrieve the top-k joined tuples that maximize an aggregate score, such as the average score.

A spatial join retrieves pairs of spatial objects that satisfy a spatial predicate, such as retrieving all intersecting pairs of objects or all pairs of objects within a certain distance from each other. As an example, given a set of all restaurants and a set of all hotels in Oslo, we would like to find pairs of restaurants and hotels that are within 1000 meters from each other for an overnight visit and dinner. The amount of restaurants and hotels within 1000 meters of each other will result in too many pairs to work with, so we would like to limit our search to highly rated pairs. We can make this top-k query, where we would like to find the 10 pairs of restaurants and hotels with the best total rating. Highly rated locations can easily be found, but the most highly rated locations are not necessarily found in close proximity to each other. We can find many pairs of locations that are close to each other, but we may not find the pairs with the best ratings quickly. An efficient solution must be able to use both their locations and their ratings to answer the query.

A spatial join can be an expensive operation because spatial datasets can be complex and very large. Spatial indexing data structures such as R-trees were created to increase the performance of spatial queries, and can be used to perform efficient spatial joins. By adapting methods from ranked joins, it is possible to answer top-k spatial joins even more efficiently by only evaluating the parts of spatial joins that are necessary to compute the answer. 

R-trees were traditionally designed to optimize I/O performance. With hardware trends such as reduced I/O costs, increasing memory sizes, multicore processors and graphics processors becoming commodity hardware, new methods are invented to fully utilize the hardware. One way to increase performance is by exploiting parallelism, which is enabled by multicore processors, distributed computing and graphics processors. However, traditional methods often require careful redesign to exploit parallelism, or entirely new methods must be designed.

The trend of General Purpose computing on Graphics processors (GPGPU) is enabled by programmable graphics processors using APIs such as CUDA\@. GPGPU presents an opportunity to achieve massively parallel computation, but not without limitations. The architecture of GPUs and CPUs are dissimilar in ways that require special application design to be able to fully utilize the resources of a GPU, and not all tasks are suited for GPGPU.\@ The use of GPGPU in certain database operators has been studied and has been shown to have speedups compared to CPU for operations such as relational joins~\cite{he2008relational} and spatial joins~\cite{yampaka2012spatial}. We would like to find out if similar speedups can be achieved for top-k spatial joins using GPGPU.\@

Top-k queries, top-k joins and spatial joins have been extensively studied. However, joins considering both spatial and score attributes at the same time have received limited attention~\cite{qi2013efficient}. 

\section{Related Work}

%\begin{figure}[h]
%    \centering
%    \includegraphics[scale=0.5]{ntnu}
%    \caption{Norwegian Institute of Science and Technology}
%    \label{fig:ntnu}
%\end{figure}