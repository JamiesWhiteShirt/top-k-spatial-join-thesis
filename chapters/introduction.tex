The spatial join operator retrieves pairs of objects that satisfy a spatial predicate, for example by finding objects that intersect each other. In many applications objects have both spatial attributes and non-spatial attributes. As an example, an application such as Google Maps provides points of interest and businesses with reviews, user ratings and various other metadata. Spatial objects are also recorded in scientific fields such as atmospheric, oceanographic and environmental sciences with measurements of several attributes such as temperature, pressure and seismic activity.

Object attributes can be used to derive a ranking, which often is used to perform a top-k query. A top-k query retrieves the k objects with the best ranking. A top-k join query is performed on ranked inputs in order to retrieve the top-k joined tuples which maximize an aggregate function. An example top-k join query can be expressed in the following SQL snippet, where objects are joined by simple attribute equality and ranked according to the sum of their scores:

\begin{verbatim}
SELECT R.id, S.id, R.score + S.score AS aggregate_score
FROM R, S
WHERE R.attribute = S.attribute
ORDER BY R.score + S.score DESC
LIMIT k
\end{verbatim}

Top-k queries, top-k joins and spatial joins have been extensively studied. However, joins considering both spatial and score attributes at the same time have received limited attention~\cite{qi2013efficient}. As an example, given a set of restaurants and a set of hotels in Oslo with ratings from 1 to 10, we would like to find the 10 pairs of restaurants and hotels that are within 500 meters from each other with the best total rating. We can easily find highly rated locations and the locations that are close to each other, but the challenge is how to use both their locations and their ratings to answer the query efficiently.

A spatial join can be an expensive operation because spatial datasets can be complex and very large. Spatial indexing data structures such as R-trees are often used to increase performance. These were traditionally designed to optimize I/O performance. With hardware trends such as reduced I/O costs, increased memory sizes and multicore processors, another way to increase performance is by exploiting parallelism. However, applications require careful design to be able to exploit parallelism, and new solutions are often required to properly utilize the processing power that is available.

The trend of General Purpose computing on Graphics processors (GPGPU) is supported by programmable graphics processors using APIs such as CUDA\@. GPGPU can be used to achieve massively parallel computation. The use of GPGPU in certain database operators has been studied and shown to have speedups compared to CPU in operators such as relational joins~\cite{he2008relational} and spatial joins~\cite{yampaka2012spatial}.

This project examines the feasibility of implementing a top-k spatial join using GPGPU\@.

\section{Related Work}

%\begin{figure}[h]
%    \centering
%    \includegraphics[scale=0.5]{ntnu}
%    \caption{Norwegian Institute of Science and Technology}
%    \label{fig:ntnu}
%\end{figure}